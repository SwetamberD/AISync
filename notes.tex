%%%%%%%%%%%%%%%%%%%%%%%%%%%%%%%%%%%%%%%%%
% Short Sectioned Assignment
% LaTeX Template
% Version 1.0 (5/5/12)
%
% This template has been downloaded from:
% http://www.LaTeXTemplates.com
%
% Original author:
% Frits Wenneker (http://www.howtotex.com)
%
% License:
% CC BY-NC-SA 3.0 (http://creativecommons.org/licenses/by-nc-sa/3.0/)
%
%%%%%%%%%%%%%%%%%%%%%%%%%%%%%%%%%%%%%%%%%

%----------------------------------------------------------------------------------------
%	PACKAGES AND OTHER DOCUMENT CONFIGURATIONS
%----------------------------------------------------------------------------------------

\documentclass[paper=a4, fontsize=12pt]{scrartcl} % A4 paper and 11pt font size

\usepackage[T1]{fontenc} % Use 8-bit encoding that has 256 glyphs
\usepackage{fourier} % Use the Adobe Utopia font for the document - comment 
%this line to return to the LaTeX default
\usepackage[english]{babel} % English language/hyphenation
\usepackage{amsmath,amsfonts,amsthm} % Math packages
\usepackage[a4paper, total={6.5in, 9in}]{geometry}
\font\myfont=cmr12 at 14pt
\usepackage{sectsty} % Allows customizing section commands
\allsectionsfont{\centering \normalfont\scshape} % Make all sections centered, 
%the default font and small caps
\date{}
\usepackage{graphicx}
\usepackage{enumitem}
\date{}
\title{Asymmetry induced synchronization in FitzHugh Nagumo oscillators}
\begin{document}
    \maketitle
\section*{Asymmetry induced Synchronization (AISync)}
\begin{itemize}
    \item Stable synchronization may be observed in networks of non-identical 
    oscillators because of (and not despite) the difference between the 
    oscillators.This phenomenon is named AISync.
    \item AISync mediated by delay-coupling is observed in networks of 
    Stuart-Landau oscillators.
    \item  Possible in networks of non-identical oscillator with arbitrary 
    differences, provided that they admit {\it at least} one common orbit when 
    coupled.
    \item The idea is oscillator heterogeneity can stabilize an otherwise 
    unstable state of complete synchronization on representative networks.
    \item In other words, AISync refers to the situation wherein identically 
    coupled oscillators synchronize stably only when the oscillator themselves 
    are non-identical.
\end{itemize}

\noindent \textbf{Conditions for AISyc}
	\begin{enumerate}[label=(\Alph*)]
		\item No asymptotically stable synchronization for homogeneous oscillators. 
		
		\item There is a heterogeneous configuration which results in stable asymptotically stable synchronization. 
	\end{enumerate}

\noindent \textbf{Some technical points}\\
{\it may not be necessary for the preliminary studies}\\
{\it Generalized master stability analysis}

\begin{enumerate}[label=(\alph*)]
	\item time delay in coupling (could be the key ingredient for AISync)
	\item finding the finest simultaneous block diagonalization.
\end{enumerate}


\section*{FitzHugh Nagumo (FTH) oscillator}
\begin{itemize}
    \item Two identically coupled FHN oscillator with one or two different 
    time delays.
    \item For two time dealay $\rightarrow$ extreme events
    \item Why should we see AISync in a network of FHN oscillators?
    \begin{enumerate}[label=(\alph*)]
    \item Single delays -- synchronization is observed (common orbit exists)
    \item Two delays -- trajectories converge
    \end{enumerate}
\end{itemize}

\textbf{Questions:}
\begin{itemize}
    \item Can we observe AISync in coupled FHN oscillators? 
    \item What network structure to choose?
    \item How about on a ring network with fractal connectivities?
    \item What happens on a network with non-symmetric structure? 
    \item Two delays may generate extreme events. What does it translates to 
    if AISync is present?
\end{itemize}

\section*{Next steps}
\begin{enumerate}
	\item Choose a version of the FTH oscillator to work with including parameter values. 
	\item The basic idea is that ``all kinds of " oscillators under consideration should admit a common orbit. Here, we will take just two kinds of oscillators -- this means that consider oscillators with two different sets of parameters. 
	\item Choose a simple symmetric network.
	\item What kind of coupling are we going to use? Possibilities are -- diffusive coupling, adjacency matrix coupling, and time-delay coupling  
	\item Challenge - Generalized master stability analysis using the finest simultaneous block diagonalization.
	\item The previous step will give maximal transverse Lyapunov exponent. Synchronization happens if this exponent is negative. 

\end{enumerate}

\section{Introduction}
FitzHugh Nagumo Oscillator:
\begin{align}
\frac{du}{dt} &= - w - u(u-1)(u-\lambda) \nonumber \\
\frac{dw}{dt} &= \epsilon(u-\delta w)
\end{align}

Properties:
\begin{itemize}
    \item Limit cycle exists for $\epsilon\delta + \lambda\leq0$.
    \item Stable limit cycle for $\epsilon > 0$ and $\delta < 
    \frac{\lambda}{\epsilon}$
    \item Hopf bifurcation occurs when $\epsilon\delta + \lambda = 0$
\end{itemize}
\section{Procedure}
\begin{itemize}
    \item We can introduce an asymmetry parameter $h$ such that $\delta 
    \rightarrow \delta - \frac{h}{\epsilon}$ and $\lambda \rightarrow 
    \lambda + h$.
    \item Therefore, we consider two kinds of FHN oscillator with parameters 
    $(\delta,\epsilon, \lambda)$ and $(\delta', \epsilon, \lambda')$ with 
    asymmetry parameter $h$ such that   \begin{align*}
    \delta' &= \delta - \frac{h}{\epsilon} \\
    \lambda'  &=\lambda + h
    \end{align*}
    \item But we have to be careful about the choice of parameters so that 
    conditions for stable limit cycle is met.
    \item 
\end{itemize}
\end{document}